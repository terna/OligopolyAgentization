\documentclass[9pt]{beamer}


\mode<presentation>
{
\usetheme{Singapore} %use default if problems or  Singapore or 
  % or ... https://deic-web.uab.cat/~iblanes/beamer_gallery/index_by_theme.html
\usefonttheme{serif}  
}
\setbeamertemplate{footline}[frame number]
\usepackage{booktabs}
\usepackage{color}

\usepackage[english]{babel}
% or whatever

\usepackage[latin1]{inputenc}
% or whatever

\usepackage{times}
\usepackage[T1]{fontenc}
% Or whatever. Note that the encoding and the font should match. If T1
% does not look nice, try deleting the line with the fontenc.
\usepackage[super]{nth}
\usepackage{xcolor}
\usepackage{relsize} %large math 

\usepackage{graphicx} % to insert the logo

\usepackage{hyperref}
\hypersetup{
  colorlinks   = true, %Colours links instead of ugly boxes
  urlcolor     = blue, %Colour for external hyperlinks
  linkcolor    = blue, %Colour of internal links
  citecolor   = red %Colour of citations
}

\usepackage[font=scriptsize,skip=1pt]{caption}

\setbeamertemplate{caption}[numbered]

\title [Oligopolistic ABM] % (optional, use only with long paper titles)
{From a theoretical oligopolistic model to a generative agent-based simulation}

\author[] % (optional, use only with lots of authors)
{P.~\href{https://terna.to.it}{Terna}\inst{1~2} \and M.~Mazzoli\inst{3} \and M.~Morini\inst{4~5}  }
% - Give the names in the same order as the appear in the paper.
% - Use the \inst{?} command only if the authors have different
%   affiliation.


\institute[] % (optional, but mostly needed)
{
  \inst{1}%
 University of Torino, Italy
  \and
 \inst{2}%
  Fondazione Collegio Carlo Alberto, Honorary Fellow, Italy
 \and
  \inst{3}%
 University of Genova, Italy 
  \and
  \inst{4}%
  Credimi S.p.A., Milano, Italy
  \and
  \inst{5}%
 Ronin Institute, Montclair, New Jersey, US
  }
% - Use the \inst command only if there are several affiliations.
% - Keep it simple, no one is interested in your street address.

\date[] % (optional, should be abbreviation of conference name)
{\href{http://proteusfoundationseries.org/event/first-international-workshop-on-agentization-rendering-conventional-models-with-agent-based-computing/}{Agentization}---September 15-17, 2021}

\begin{document}


%%%%%%%%%%%%%%%%%%%%%%%%%%%%%%%%%%%%%%%%%%%%%%%%%%%%%%%%%
\begin{frame}
\noindent\makebox[\linewidth]{\rule{\paperwidth}{0.4pt}}

\titlepage

\end{frame}

%%%%%%%%%%%%%%%%%%%%%%%%%%%%%%%%%%%%%%%%%%%%%%%%%%%%%%%%%
\begin{frame}{Outline}

  \tableofcontents
  % You might wish to add the option [pausesections]
\end{frame}

%%%%%%%%%%%%%%%%%%%%%%%%%%%%%%%%%%%%%%%%%%%%%%%%%%%%%%%%%
\section{A book on \emph{Rethinking Macroeconomics with Endogenous Market Structure}}

\subsection{Starting questions}

%%%%%%%%%%%%%%%%%%%%%%%%%%%%%%%%%%%%%%%%%%%%%%%%%%%%%%%%%
\begin{frame}{Starting questions}

%\footnote{mazzoli_morini_terna_2019}

\begin{columns}[T]
\begin{column}{.6\textwidth}
\begin{block}{}
% Your text here

\begin{itemize}

\item
Do entry, exit and changes in market structure affect the macroeconomy?

\item Is there a link between the strategic interactions among oligopolistic firms and the macroeconomic equilibrium?

\end{itemize}

\smallskip
\small
This questions are certainly not trivial in modern economies, where large oligopolistic firms play a relevant role and so many meetings among statesmen have the explicit scope of promoting contracts for some large and important firms of their countries. 

However, surprisingly enough, the most popular theoretical models in the modern macroeconomic literature hardly see any explicit formalization for the macroeconomic effects of changes in market structure, entry, exit and strategic interactions among oligopolists.
     
\end{block}
\end{column}

 \begin{column}{.4\textwidth}
 \begin{block}{}
% Your image included here
 \includegraphics[scale=0.25]{cover.png}
  \end{block}
  \end{column}
    
\end{columns}


\smallskip

\noindent\rule{8cm}{0.4pt}
\scriptsize

Mazzoli, M., Morini, M., and Terna, P. 2019. \emph{Rethinking Macroeconomics with Endogenous Market Structure}. Cambridge University Press.

\end{frame}

%%%%%%%%%%%%%%%%%%%%%%%%%%%%%%%%%%%%%%%%%%%%%%%%%%%%%%%%%

\section{Theoretical analysis}

\subsection{Main assumptions}

%%%%%%%%%%%%%%%%%%%%%%%%%%%%%%%%%%%%%%%%%%%%%%%%%%%%%%%%%
\begin{frame}{Interactions among oligopolistic firms}

We introduce a new macromodel where entry, exit and strategic interactions among oligopolistic firms are explicitly formalized and may generate macroeconomic fluctuations. 

About macroeconomic impact of business formation we refer to Gabaix (2011). His ``granular hypothesis'' was initially studied by Jaimovich and Rebelo (2009). 

\bigskip
\bigskip
\bigskip
\bigskip
\bigskip


\noindent\rule{8cm}{0.4pt}
\scriptsize


Gabaix, X. 2011. The granular origins of aggregate fluctuations. \emph{Econometrica}, \textbf{79}(3), 733?72.

Jaimovich, N., and Rebelo, S. 2009. Can news about the future drive the
business cycle? \emph{American Economic Review}, \textbf{99}(4), 1097--118.

\end{frame}


%%%%%%%%%%%%%%%%%%%%%%%%%%%%%%%%%%%%%%%%%%%%%%%%%%%%%%%%%
\begin{frame}{Aggregate demand}

The microfounded optimization problem of the heterogeneous consumers with the same preferences but different budget constraint (depending on wether they are workers, new entrants or incumbent entrepreneurs) yields the following aggregate demand:

\begin{eqnarray}
D(\cdot )_{t} &=&\frac{\Omega (R_{t})}{P_{t}}\{A_{t}+((1+r_{t})(1+\iota
)^{-1}\sum_{i=0}^{\infty }[(1+E\left( r_{t+i}\right) (1+\iota )]^{-i}\cdot 
\notag \\
&&\cdot E(n_{t+i}(W_{t+i}+h_{t+i}^{e}\Pi _{t+i}^{e}+h_{t+i}^{in}\Pi
_{t+i}^{in}))\}  \label{per capita distrib cons_1}
\end{eqnarray}

\small

$\Pi_{t+i}^{in}$ and $\Pi _{t+i}^{e}$ are the nominal profits of the incumbent and new entrants entrepreneurs;

$r_{t}$ is  the real interest rate at time $t$; $R_{t}$ is the nominal interest rate on the financial asset at time $t$ (controlled by the central bank); $\iota$ is the <<core>> inflation rate, assumed to be constant under a given monetary policy regime;

$W_{t+i}$ the nominal wage at time $t+i$; $n_{t+i}$ the total number of employed individuals at time $t+i$,

$h_{t+i}^{in}$ and $h_{t+i}^{e}$ the portion of incumbent entrepreneurs and new entrant over the total labor force;

$Pi _{t+i}$ the price level emerging in the oligopolistic industrial sector (which is also the aggregate price level since we have an indifferentiated good);

$\Omega_{t+i}$ a monotonically increasing function in the nominal interest rate.

\end{frame}


%%%%%%%%%%%%%%%%%%%%%%%%%%%%%%%%%%%%%%%%%%%%%%%%%%%%%%%%%
\begin{frame}{Outputs}

The output of each firm is given by the production function, where $\psi _{i,t}$ is a positive or negative shock:

\begin{equation}
\varphi _{i,t}=\Lambda L_{i,t}^{\alpha }+\psi _{i,t}
\end{equation}

Summing up, with $P_t$ the aggregate price level (as weighted average of firms' prices) and having $\psi _{i,t}$ a zero average:

\begin{equation}
Y_{t}=P_{t}\Lambda \overset{H_{t}}{\underset{i=1}{\sum }}(L_{i,t}^{\alpha}
\end{equation}

\small
$H_{t}$ is is the total number of oligopolistic firms operating at time t;

$\Lambda$ the usual constant parameter capturing technology shocks;

\end{frame}




\end{document}