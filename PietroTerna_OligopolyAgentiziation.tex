\documentclass[9pt]{beamer}


\mode<presentation>
{
\usetheme{Singapore} %use default if problems or  Singapore or 
  % or ... https://deic-web.uab.cat/~iblanes/beamer_gallery/index_by_theme.html
\usefonttheme{serif}  
}
\setbeamertemplate{footline}[frame number]
\usepackage{booktabs}
\usepackage{color}

\usepackage[english]{babel}
% or whatever

\usepackage[latin1]{inputenc}
% or whatever

\usepackage{times}
\usepackage[T1]{fontenc}
% Or whatever. Note that the encoding and the font should match. If T1
% does not look nice, try deleting the line with the fontenc.
\usepackage[super]{nth}
\usepackage{xcolor}
\usepackage{relsize} %large math 

\usepackage{graphicx} % to insert the logo

\usepackage{hyperref}
\hypersetup{
  colorlinks   = true, %Colours links instead of ugly boxes
  urlcolor     = blue, %Colour for external hyperlinks
  linkcolor    = blue, %Colour of internal links
  citecolor   = red %Colour of citations
}

\usepackage[font=scriptsize,skip=1pt]{caption}

\setbeamertemplate{caption}[numbered]

\title [Oligopolistic ABM] % (optional, use only with long paper titles)
{From a theoretical oligopolistic model to a generative agent-based simulation}

\author[] % (optional, use only with lots of authors)
{P.~\href{https://terna.to.it}{Terna}\inst{1~2} \and M.~Mazzoli\inst{3} \and M.~Morini\inst{4~5}  }
% - Give the names in the same order as the appear in the paper.
% - Use the \inst{?} command only if the authors have different
%   affiliation.


\institute[] % (optional, but mostly needed)
{
  \inst{1}%
 University of Torino, Italy
  \and
 \inst{2}%
  Fondazione Collegio Carlo Alberto, Honorary Fellow, Italy
 \and
  \inst{3}%
 University of Genova, Italy 
  \and
  \inst{4}%
  Credimi S.p.A., Milano, Italy
  \and
  \inst{5}%
 Ronin Institute, Montclair, New Jersey, US
  }
% - Use the \inst command only if there are several affiliations.
% - Keep it simple, no one is interested in your street address.

\date[] % (optional, should be abbreviation of conference name)
{\href{http://proteusfoundationseries.org/event/first-international-workshop-on-agentization-rendering-conventional-models-with-agent-based-computing/}{Agentization}---September 15-17, 2021}

\begin{document}

%%%%%%%%%%%%%%%%%%%%%%%%%%%%%%%%%%%%%%%%%%%%%%%%%%%%%%%%%
\begin{frame}

\titlepage

\end{frame}

%%%%%%%%%%%%%%%%%%%%%%%%%%%%%%%%%%%%%%%%%%%%%%%%%%%%%%%%%
\begin{frame}{Outline}

  \tableofcontents
  % You might wish to add the option [pausesections]
\end{frame}

%%%%%%%%%%%%%%%%%%%%%%%%%%%%%%%%%%%%%%%%%%%%%%%%%%%%%%%%%
\section{The model}

\subsection{Introduction}

%%%%%%%%%%%%%%%%%%%%%%%%%%%%%%%%%%%%%%%%%%%%%%%%%%%%%%%%%
\begin{frame}{Introduction}

\begin{itemize}

\item
A micro-based model of interacting agents, following plausible behavioral rules into a world where the Covid-19 epidemic is affecting the actions of everyone. 
\item
The model works with: 

\begin{enumerate}[i]
\item infected agents categorized as symptomatic or asymptomatic and 
\item the places of contagion specified in a detailed way, thanks to agent-based modeling capabilities. 
\end{enumerate}

 \item
The \textcolor{red}{infection transmission} is related to three factors: the infected person's characteristics and those of the susceptible one, plus those of the space in which a contact occurs.

\end{itemize}
\end{frame}

%%%%%%%%%%%%%%%%%%%%%%%%%%%%%%%%%%%%%%%%%%%%%%%%%%%%%%%%%

\bibliographystyle{plainnatmm}
\bibliography{bibliografiaGenerale}


\end{document}